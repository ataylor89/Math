\documentclass{article}
\usepackage{amsmath, amssymb, amsthm}
\newtheorem{lemma}{Lemma}
\newtheorem{theorem}{Theorem}
\title{Integrating a polynomial}
\author{Andrew Taylor}
\date{March 5 2022}
\begin{document}
\maketitle

\begin{lemma}
Let n and m be natural numbers. Then

\begin{equation*}
\sum_{k=1}^{n} k^{m} = \frac{n^{m+1} + \displaystyle \sum_{k=1}^{n} \left( \binom{m+1}{2} k^{m-1} (-1)^2 +  \binom{m+1}{3} k^{m-2} (-1)^3 + ... +  \binom{m+1}{m+1} k^{0} (-1)^{m+1} \right) }{m+1}
\end{equation*}
\end{lemma}

\begin{proof}
Consider the sum 

\begin{equation*}
\sum_{k=1}^{n} \left( k^{m+1} - (k-1)^{m+1} \right)
\end{equation*}

We know that this sum telescopes, which is to say, all but the first and last terms cancel.

\begin{equation*}
\sum_{k=1}^{n} \left( k^{m+1} - (k-1)^{m+1} \right) = n^{m+1} - 0^{m+1} = n^{m+1}
\end{equation*}

We can expand the binomial $(k-1)^{m+1}$ and get

\begin{align*}
n^{m+1} &= \sum_{k=1}^{n} \left( k^{m+1} - (k-1)^{m+1} \right) \\
&= \sum_{k=1}^{n} \left( k^{m+1} - \left( \binom{m+1}{0} k^{m+1} (-1)^{0} + \binom{m+1}{1} k^{m} (-1)^{1} + ... + \binom{m+1}{m+1} k^{0} (-1)^{m+1} \right) \right) \\
&= \sum_{k=1}^{n} -1 \left( \binom{m+1}{1} k^{m} (-1)^{1} + \binom{m+1}{2} k^{m-1} (-1)^{2} + ... + \binom{m+1}{m+1} k^{0} (-1)^{m+1} \right)  
\end{align*}

We can solve the above equation for the term $\displaystyle \sum_{k=1}^{n} k^{m}$

\begin{equation*}
\sum_{k=1}^{n} \binom{m+1}{1} k^{m} = n^{m+1} + \displaystyle \sum_{k=1}^{n} \left( \binom{m+1}{2} k^{m-1} (-1)^{2} + ... + \binom{m+1}{m+1} k^{0} (-1)^{m+1} \right) 
\end{equation*}

We can divide both sides by $m+1$.

\begin{equation*}
\sum_{k=1}^{n} k^{m} = \frac{n^{m+1} + \displaystyle \sum_{k=1}^{n} \left( \binom{m+1}{2} k^{m-1} (-1)^2 +  \binom{m+1}{3} k^{m-2} (-1)^3 + ... +  \binom{m+1}{m+1} k^{0} (-1)^{m+1} \right) }{m+1}
\end{equation*}

\end{proof}

\begin{lemma}
Let $b_{n}$ be the sequence 

\begin{equation*}
b_{n} = \frac{1}{m+1} \displaystyle \sum_{k=1}^{n} \left( \binom{m+1}{2} k^{m-1} (-1)^2 +  \binom{m+1}{3} k^{m-2} (-1)^3 + ... +  \binom{m+1}{m+1} k^{0} (-1)^{m+1} \right) 
\end{equation*}

for some $m \in \mathbb{N}$. Then there exists a $C \in \mathbb{R}$ such that

\begin{equation*}
-C \, n^{m} <= b_{n} <= C \, n^{m}
\end{equation*}

\end{lemma}

\begin{proof}
Define the polynomial $P(k)$ as 

\begin{equation*}
P(k) = \binom{m+1}{2} k^{m-1} (-1)^2 +  \binom{m+1}{3} k^{m-2} (-1)^3 + ... +  \binom{m+1}{m+1} k^{0} (-1)^{m+1}
\end{equation*}

Now we can write 

\begin{equation*}
b_{n} = \frac{1}{m+1} \displaystyle \sum_{k=1}^{n} P(k)
\end{equation*}

Let $X$ be the product of coefficients in $P(k)$ and let $C = m * |X|$ \\

Define the polynomial $Q(k)$ as

\begin{equation*}
Q(k) = C \, k^{m-1}
\end{equation*}

We have

\begin{equation*}
|P(k)| \le |Q(k)|
\end{equation*}

Let $c_{n}$ be the sequence defined by

\begin{equation*}
c_{n} = \frac{1}{m+1} \displaystyle \sum_{k=1}^{n} Q(k)
\end{equation*}

We know that $|b_{n}| \le |c_{n}|$ since $|P(k)| \le |Q(k)|$ \\

Furthermore we know that

\begin{equation*}
\left| \sum_{k=1}^{n} Q(k) \right| \le \left| Cn^{m} \right|
\end{equation*}

Thus

\begin{equation*}
|c_{n}| \le \left| \frac{Cn^{m}}{m+1} \right|
\end{equation*}

By the transitive property we get

\begin{equation*}
\frac{-Cn^{m}}{m+1} \le b_{n} \le \frac{Cn^{m}}{m+1} 
\end{equation*}

\end{proof}

\begin{lemma}
Let n and m be natural numbers. Let b be a real number.

\begin{equation*}
\lim_{n \to \infty} \frac{b^{m+1}}{n^{m+1}} \sum_{k=1}^{n} k^{m} = \displaystyle \frac{b^{m+1}}{m+1}
\end{equation*}
\end{lemma}

\begin{proof}

Define A and B as 

\begin{align*}
A &= \displaystyle \frac{n^{m+1}}{m+1} \\
B &= \frac{1}{m+1} \displaystyle \sum_{k=1}^{n} \left( \binom{m+1}{2} k^{m-1} (-1)^2 +  \binom{m+1}{3} k^{m-2} (-1)^3 + ... +  \binom{m+1}{m+1} k^{0} (-1)^{m+1} \right) 
\end{align*}

Then

\begin{align*}
\lim_{n \to \infty} \frac{b^{m+1}}{n^{m+1}} \sum_{k=1}^{n} k^{m} &= \lim_{n \to \infty} \frac{b^{m+1}}{n^{m+1}} (A + B) \\
&= \lim_{n \to \infty} \frac{b^{m+1}}{n^{m+1}} A + \lim_{n \to \infty} \frac{b^{m+1}}{n^{m+1}} B \\
\end{align*}

Now we can solve each limit separately.

\begin{align*}
\lim_{n \to \infty} \frac{b^{m+1}}{n^{m+1}} A &=  \lim_{n \to \infty} \frac{b^{m+1}}{n^{m+1}} \frac{n^{m+1}}{m+1} \\
&=  \frac{b^{m+1}}{m+1} 
\end{align*}

We know from lemma 2 that there exists a $C \in \mathbb{R}$ such that 

\begin{equation*}
|B| \le |C \, n^{m}|
\end{equation*}

It follows that

\begin{equation*}
\left| \frac{b^{m+1}}{n^{m+1}} B \right| \le \left| \frac{b^{m+1}}{n^{m+1}} Cn^{m} \right|
\end{equation*}

We also know that

\begin{align*}
\lim_{n \to \infty} \frac{b^{m+1}}{n^{m+1}} C n^{m}  = \lim_{n \to \infty} \frac{C b^{m+1}}{n} = 0
\end{align*} 

Thus, by the pinching theorem:

\begin{align*}
\lim_{n \to \infty} \frac{b^{m+1}}{n^{m+1}} B = 0
\end{align*}

This gives us

\begin{align*}
\lim_{n \to \infty} \frac{b^{m+1}}{n^{m+1}} \sum_{k=1}^{n} k^{m} &= \lim_{n \to \infty} \frac{b^{m+1}}{n^{m+1}} (A + B) \\
&= \lim_{n \to \infty} \frac{b^{m+1}}{n^{m+1}} A + \lim_{n \to \infty} \frac{b^{m+1}}{n^{m+1}} B \\
&= \frac{b^{m+1}}{m+1} + 0 \\
&= \frac{b^{m+1}}{m+1} 
\end{align*}

\end{proof}

\begin{theorem}
Let $m \in \mathbb{N}$ and let $f: \mathbb{R} \to \mathbb{R}$ be defined by $f(x) = x^{m}$. Then
\begin{equation*}
\int_{0}^{b} f(x) \,dx  = \frac{b^{m+1}}{m+1}
\end{equation*}
\end{theorem}

\begin{proof}
We can form an equation using the Riemann integral.

\begin{align*}
\int_{0}^{b} f(x) \,dx &= \lim_{n \to \infty} \frac{b - 0}{n} \sum_{k=1}^{n} f(0 + \frac{k(b-0)}{n}) \\
&= \lim_{n \to \infty} \frac{b}{n} \sum_{k=1}^{n} \left( \frac{kb}{n} \right) ^{m} \\
&= \lim_{n \to \infty} \frac{b^{m+1}}{n^{m+1}} \sum_{k=1}^{n} k^{m}
\end{align*}

We can substitute the result from lemma 3, which gives us

\begin{align*}
\int_{0}^{b} f(x) \,dx &= \lim_{n \to \infty} \frac{b^{m+1}}{n^{m+1}} \sum_{k=1}^{n} k^{m} \\
&= \frac{b^{m+1}}{m+1}
\end{align*}

\end{proof}

\end{document}