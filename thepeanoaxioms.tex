\documentclass{article}
\usepackage{amsmath, amssymb, amsthm}
\newtheorem{axiom}{Axiom}
\newtheorem{definition}{Definition}
\newtheorem{lemma}{Lemma}
\newtheorem{theorem}{Theorem}
\newtheorem{remark}{Remark}
\newcommand{\inc}[1]{#1\!+\!+}
\newcommand{\incp}[1]{(#1\!+\!+)}
\title{The Peano axioms}
\author{Andrew Taylor}
\date{July 16 2022}
\begin{document}
\maketitle

\section{The Peano axioms and the successor function}

\begin{definition}
An axiom is a fundamental rule that does not require proof.
\end{definition}

Axioms are essential to mathematics. When we construct a number system, we can start by defining the fundamental rules that govern the number system. Then we can see whether there are any number systems that obey these rules. \\

In this paper, we will define the axioms that govern the natural numbers. We call these axioms the Peano axioms. We will use the formulation of the Peano axioms given by Terence Tao in his book Analysis-1. \\

There's a difference between creating rules for the natural numbers and constructing the natural numbers. We may wait a while to construct the natural numbers. First we will give the rules for the natural numbers, and discover all the properties and theorems that derive from these rules.

\begin{definition}
A set is any unordered collection of objects. A set is also an object. Thus a set can be an element of another set.
\end{definition}

Above is the definition that Terence Tao gives in his book Analysis-1.

\begin{definition}
Let X and Y be sets. A function from X to Y is a correspondence between X and Y such that for every $x \in X$, there is exactly one $y \in Y$ that corresponds to x. 
\end{definition}

This is a very useful definition of a function. It is simply put, and uses the concept of a correspondence.

\begin{definition}
Let $\mathbb{N}$ denote the set of natural numbers. We will assume the natural numbers exist, without constructing a model for the natural numbers.
\end{definition}

It's common to construct the natural numbers using set theory. After constructing the natural numbers, one must show that they agree with the Peano axioms. This will come later. For now we focus on the Peano axioms.

\begin{axiom}
0 is a natural number.
\end{axiom}

We have just created a natural number axiomatically. It's our first natural number. We call it zero. \\

To create more natural numbers, we need to define the successor operation.

\begin{definition}
Let $S(n)$ be the successor operation, where $S(n)$ is the successor of n for every $n \in \mathbb{N}$. The notation $\inc{n}$ is a shorthand for $S(n)$. The successor operation can also be called the increment operation.
\end{definition}

It's not enough to define the successor operation. We also want to place constraints on what a successor to a natural number can be. For example, we know nothing about the successor to 0. Is it pi? Is it -1? A second axiom is needed to tell us more about the successor of a natural number.

\begin{axiom}
If n is a natural number, then the successor to n is a natural number. That is, $S(n)$ is a natural number. We can also write $\inc{n}$ is a natural number, since these statements are the same.
\end{axiom}

With these two axioms, it is possible for there to only be one natural number. Consider the example where the successor to zero is zero. Then there would be only one natural number. Since we want to create a number system where there is more than one number, we need to add more axioms.

\begin{axiom}
0 is not the successor of any natural number.
\end{axiom}

This third axiom guarantees that there are at least two natural numbers. Let's call the second natural number one. But suppose the successor to one is one? Then we would only have two natural numbers. We want a number system that has more than two numbers, so once again, we must add another axiom.

\begin{axiom}
If $n \neq m$, then $S(n) \neq S(m)$, or in other words, $\inc{n} \neq \inc{m}$.
\end{axiom}

Now our number system is infinite. The fourth Peano axiom prevents any cycles or wrap-arounds. The successor to 1 cannot be 1, because then 0 and 1 would have the same successor. 

\begin{axiom}
Let $P(n)$ be a property that is true for a natural number n. Suppose that $P(0)$ is true, and suppose that whenever $P(n)$ is true, $P(\inc{n})$ is also true. Then $P(n)$ is true for every natural number n.
\end{axiom}

This last axiom allows us to do induction. It's important to realize why. \\

Without the fifth Peano axiom, the natural numbers could have two heads. For example, we could have 0 and 0-prime. Then we would have one sequence $0, 1, 2, 3, ...$ and another sequence $0', 1', 2', 3', ...$. The second sequence would have the same properties as the first sequence, and this would make it impossible to do induction. But with the fifth Peano axiom in place, we are able to do induction. The fifth Peano axiom also lets us prove that 0 is the only natural number that is not the successor of another natural number. \\

Let's proceed with writing some theorems about the natural numbers and proving them.

\begin{theorem}
Every natural number is either zero or the successor of a natural number.
\end{theorem}

\begin{proof}
We will use the principle of induction stated in the fifth Peano axiom. Let $P(n)$ signify that n is either zero or the successor of a natural number. This property is clearly true for $n = 0$. Suppose inductively that it is true for n. Then it is also true for $\inc{n}$, because $\inc{n}$ is the successor of n. This closes our induction. Thus $P(n)$ is true for every natural number n. Therefore a natural number is either zero or the successor of a natural number.
\end{proof}

\begin{theorem}
The successor operation $S(n)$ is a function from $\mathbb{N}$ to $\mathbb{N}\setminus\{0\}$.
\end{theorem}

\begin{proof}
Let $n \in \mathbb{N}$. By the second Peano axiom, there corresponds to n exactly one $m \in \mathbb{N}$ such that $S(n) = m$. By the third Peano axiom, $m \neq 0$. Thus $S(n)$ is a function from $\mathbb{N}\setminus\{0\}$.
\end{proof}

\begin{theorem}
The successor operation $S : \mathbb{N} \to \mathbb{N}\setminus\{0\}$ is a bijection.
\end{theorem}

\begin{proof}
We know that $S(n)$ is a surjection because the range is a subset of the domain. By the fourth Peano axiom, we know that $S(n)$ is also an injection (one-to-one). Thus $S(n)$ is a bijection.
\end{proof}

\begin{theorem}
There are infinitely many natural numbers. 
\end{theorem}

\begin{proof}
Suppose there are finitely many natural numbers.We know that $S : \mathbb{N} \to \mathbb{N}\setminus\{0\}$ is a bijection. Since the range is a subset of the domain, and the domain is a finite set, it must be the case that the domain and range are equal sets. In other words, $\mathbb{N} = \mathbb{N}\setminus\{0\}$. But this is a contradiction, since 0 is in the domain but not in the range. Thus there are infinitely many natural numbers. 
\end{proof}

\begin{remark}
To prove the next theorem we will take for granted the rules of algebra, which we have not yet derived.
\end{remark}

\begin{theorem}
There are infinitely many prime numbers.
\end{theorem}

\begin{proof}
Suppose there are finitely many prime numbers. Let $P = \{p_{1}, p_{2}, p_{3}, ... ,p_{n}\}$ be the set of prime numbers. Let $x = p_{1}p_{2}p_{3} \dotsm p_{n} + 1$. Let k be a natural number such that $1 \leq k \leq n$. Then 

\begin{align*}
\dfrac{x}{p_{k}} &= \dfrac{p_{1}p_{2}p_{3} \dotsm p_{n} + 1}{p_{k}} \\
&= \dfrac{p_{1}p_{2}p_{3} \dotsm p_{n}}{p_{k}} + \dfrac{1}{p_{k}} \\
&= y + \dfrac{1}{p_{k}}
\end{align*}

In the resulting expression, $y$ is a natural number and $\dfrac{1}{p_{k}}$ is not a natural number. Thus the sum cannot be a natural number, which means that x is not divisible by any prime. If x is not divisible by any prime, then x must be prime. But this is a contradiction, since x is not an element of P. Thus there are infinitely many primes. 
\end{proof}

\begin{remark}
Let's review. S is the successor function. The second Peano axiom makes it necessary that S is a surjection. The fourth Peano axiom makes it necessary that S is an injection. Therefore S is a bijection. The third Peano axiom tells us that the image of S is a proper subset of the natural numbers. In order for the natural numbers to be consistent with these axioms, the natural numbers have to be infinite.
\end{remark}

\section{Addition of natural numbers}

\begin{definition}
Let m be a natural number. We define $0 + m = m$. Now suppose inductively that we have defined $n + m$ for a natural number n. Then we define $(\inc{n}) + m = \ \inc{(n+m)}$. 
\end{definition}

\begin{lemma}
For any natural number n, $n + 0 = n$.
\end{lemma}

\begin{proof}
We will induct on n. In the base case where $n = 0$, we see that $0 + 0 = 0$, which follows from the definition of addition. Suppose inductively that $n + 0 = n$. Then $(\inc{n}) + \ 0 \ = \ \inc{(n+0)}$, which also follows from the definition of addition. We can substitute n for $n + 0$ and we get $(\inc{n}) + 0 = \inc{n}$, which is what we wanted. This closes our induction. Thus $n + 0 = n$ for all $n \in \mathbb{N}$.
\end{proof}

\begin{lemma}
For any natural numbers n and m, we have $n + (\inc{m}) = \inc{(n+m)}$.
\end{lemma}

\begin{proof}
We will fix m, and induct on n. In the base case, we have $0 + (\inc{m}) = \inc{m}$, which follows from our definition of addition. We can then substitute $0 + m$ for m and get $0 + (\inc{m}) = \inc{(0+m)}$. We can make this substitution because of our definition of addition. Now suppose inductively that $n + (\inc{m}) = \inc{(n+m)}$. Then $(\inc{n}) + (\inc{m}) = \inc{(n + \inc{m})}$, using our definition of addition. Now using our inductive assumption, we can substitute $\inc{(n+m)}$ into our equation, getting $(\inc{n}) + (\inc{m}) = \inc{( \inc{(n+m)})}$. Using the definition of addition, we can make another substitution, getting $(\inc{n}) + (\inc{m}) = \inc{((\inc{n}) + m)}$. This is what we wanted, and it closes our induction. Thus for any natural numbers n and m, we have $n + (\inc{m}) = \inc{(n+m)}$.
\end{proof}

\begin{theorem}
Let n and m be natural numbers. Then $n + m = m + n$. In other words, addition is commutative.
\end{theorem}

\begin{proof}
We will fix m and induct on n. For the base case we have $0 + m = m + 0$. The definition of addition tells us that $0 + m = m$. Our lemma tells us that $m + 0 = m$. Thus $0 + m = m + 0$. Now suppose inductively that $n + m = m + n$ for a natural number n. Then $\incp{n} + m = \inc{(n+m)}$ by our definition of addition. And $m + \incp{n} = \inc{(m + n)}$ by our lemma. We can substitute $n + m$ for $m + n$ because of our inductive assumption, and that gives us $\incp{n} + m = m + \incp{n}$, which is what we wanted. This closes our induction. Thus addition is commutative, and $n + m = m + n$ for all natural numbers n and m.
\end{proof}

\begin{theorem}
Let a, b, and c be natural numbers. Then $(a + b) + c = a + (b + c)$. In other words, addition is associative.
\end{theorem}

\begin{proof}
We will fix a and b, and induct on c. In the base case we have $(a + b) + 0 = a + b$ and $a + (b + 0) = a + b$. Thus $(a + b) + 0 = a + (b + 0)$. Now suppose inductively that $(a + b) + c = a + (b + c)$. We know from our lemma that $(a + b) + \inc{c} = \inc{((a + b) + c)}$. We also know from our lemma that $a + (b + \inc{c}) = a + \inc{(b + c)} = \inc{(a + (b + c))}$. By our inductive assumption, we know that $\inc{((a + b) + c)} = \inc{(a + (b + c))}$. Therefore $(a + b) + \inc{c} = a + (b + \inc{c})$. This closes our induction. Thus $(a + b) + c = a + (b + c)$ for all natural numbers a, b, and c.
\end{proof}

\end{document}