\documentclass{article}
\usepackage{amsmath, amssymb, amsthm}
\newtheorem{definition}{Definition}
\newtheorem{theorem}{Theorem}
\title{Rational numbers}
\author{Andrew Taylor}
\date{March 12 2022}
\begin{document}
\maketitle
\begin{definition}
Let A and B be sets. The Cartesian product of A and B, denoted $A \times B$, is the set
\begin{equation*}
A \times B = \{(a, b) \mid a \in A, b \in B\}
\end{equation*}
\end{definition}
\begin{definition}
Let X be a set. A relation R on X is a subset of $X \times X$. 
\\
\\
If $(a, b) \in R$, we say ``a is related to b'', and write $a \sim b$.
\end{definition}
\begin{definition}
Let X be a set. An equivalence relation R on X is a relation such that
\begin{enumerate}
\item For all $a \in X$, $(a, a) \in R$
\item For all $a, b \in X$, if $(a, b) \in R$, then $(b, a) \in R$
\item For all $a, b, c \in X$, if $(a, b) \in R$ and $(b, c) \in R$, then $(a, c) \in R$
\end{enumerate}
If $(a, b) \in R$, we say ``a is equivalent to b'', and write $a \sim b$.
\end{definition}
\begin{definition}
Let X be a set and let R be an equivalence relation on X. Let $a \in X$. The equivalence class of a, denoted $C(a)$, is defined by 
\begin{equation*}
C(a) = \{b \in X \mid b \sim a\}
\end{equation*}
\end{definition}
\begin{definition}
Let $F = \{(a, b) \mid a, b \in \mathbb{Z}\  and \  b \ne 0\}$. Let $(a, b), (c, d) \in X$. 
\\
\\
We define the relation $\sim$ on F as $(a, b) \sim (c, d)$ if $ad = bc$. 
\end{definition}
\begin{theorem}
The relation $\sim$ is an equivalence relation on F.
\end{theorem}
\begin{proof}
Let $(a, b) \in F$. We know that $(a, b) \sim (a, b)$ because $a * b = b * a$. 
\\
\\
Let $(a, b), (c, d) \in F$ such that $(a, b) \sim (c, d)$. By definition, $ad = bc$. Remember that $=$ is an equivalence relation on the integers. By the symmetric property, $bc = ad$. Since multiplication is commutative, we can write, $cb = da$. This is the equation we want. The equation $cb = da$ tells us that $(c, d) \sim (a, b)$. 
\\
\\
Let $(a, b), (c, d), (e, f) \in F$ such that $(a, b) \sim (c, d)$ and $(c, d) \sim (e, f)$. 
\\
\\
Then $ad = bc$ (1) and $cf = de$ (2). 
\\
\\
We can multiply both sides of equation (1) by f.
\begin{equation*}
adf = bcf
\end{equation*}
Now we can substitute $de$ for $cf$ in the above equation.
\begin{equation*}
adf = bde
\end{equation*}
We know that d is nonzero by definition, so we can divide both sides by $d$.
\begin{equation*}
af = be
\end{equation*}
Thus $(a, b) \sim (e, f)$. 
\\
\\
The relation $\sim$ satisfies the reflexive property, the symmetric property, and the transitive property. Therefore $\sim$ is an equivalence relation on F.
\end{proof}

\begin{definition}
We define the rational numbers $\mathbb{Q}$ as the set of equivalence classes in F determined by the equivalence relation $\sim$.
\begin{equation*}
\mathbb{Q} = \{C((a, b)) \mid (a, b) \in F\}
\end{equation*}
\end{definition}
\end{document}