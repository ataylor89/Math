\documentclass{article}
\usepackage{amsmath, amssymb, amsthm}
\title{The quadratic formula}
\author{Andrew Taylor}
\date{August 3 2022}
\begin{document}
\maketitle

We can derive the quadratic formula by solving the quadratic equation $ax^2 + bx + c = 0$ for x. \\

We solve this equation by completing the square. We assume $a \neq 0$.

\begin{align*}
ax^2 + bx + c &= 0 \\
x^2 + \dfrac{b}{a}x + \dfrac{c}{a} &= 0 \\
x^2 + \dfrac{b}{a}x + \left(\dfrac{b}{2a}\right)^2 + \dfrac{c}{a} &= \left(\dfrac{b}{2a}\right)^2 \\
\left(x + \dfrac{b}{2a}\right)^2 + \dfrac{c}{a} &= \left(\dfrac{b}{2a}\right)^2 \\
\left(x + \dfrac{b}{2a}\right)^2 &= \dfrac{b^2 - 4ac}{4a^2} \\
x + \dfrac{b}{2a} &= \dfrac{\pm \sqrt{b^2 - 4ac}}{2a} \\
x &= \dfrac{-b \pm \sqrt{b^2 - 4ac}}{2a} \\
\end{align*}

This technique (completing the square) was discovered by the Arabic mathematician al-Khwarizmi in the early 9th century CE. \\

You can see how we use the rules of algebra to get a square on the lefthand side of the equation. When we take the square root of both sides, we are left with only one term that has the variable x. \\

Our equations reveal the journey of isolating the variable x, so that only one term contains the variable x. \\

The expression $b^2 - 4ac$ is called the discriminant. When $b^2 - 4ac$ is negative, the solutions are imaginary numbers. When $b^2 - 4ac$ is nonnegative, the solutions are real numbers. \\

The only constraint on this formula is that $a \neq 0$. When $a = 0$ the equation is linear, not quadratic. \\

Thus we have discovered the quadratic formula by using algebraic operations on the quadratic equation $ax^2 + bx + c = 0$ to isolate the variable x. In other words, we have discovered the quadratic formula by completing the square. \\

The quadratic formula is \begin{align*}x = \dfrac{-b \pm \sqrt{b^2 - 4ac}}{2a}\end{align*}
\end{document}