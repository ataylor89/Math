\documentclass{article}
\usepackage{amsmath, amssymb, amsthm}
\usepackage{graphicx}
\newtheorem{lemma}{Lemma}
\newtheorem{theorem}{Theorem}
\title{Finding a formula for the sum of squares}
\author{Andrew Taylor}
\date{February 21 2022}
\begin{document}
\maketitle
\section{Introduction}
\begin{flushleft}
The goal of this paper is to derive a formula for the sum of squares. 
\end{flushleft}
\begin{flushleft}
Before working on this, we will explain what a formula is, what a sequence is, and how to add the terms of a sequence.
\end{flushleft}
\section{Formulas}
\begin{flushleft}
A formula is a function. A function is a correspondence between two sets X and Y such that each element in X is associated with exactly one element in Y.
\end{flushleft}
\begin{flushleft}
On the next page there is a graph of a function.
\end{flushleft}
\begin{figure}
\centering
\includegraphics{xsquared}
\end{figure}
\clearpage
\begin{flushleft}
The graph in the previous page illustrates the function $f(x) = x^2$. 
\end{flushleft}
\begin{flushleft}
One thing to notice in the graph is that the x and y axes both visualize the real numbers. The x axis visualizes the interval $[-10, 10]$ and the y axis visualizes the interval $[0, 100]$.
\end{flushleft}
\begin{flushleft}
The function $f$ is a correspondence between the real numbers. In other words, $f$ has the real numbers as its domain and range. We can completely describe the function by saying $f : \mathbb{R} \rightarrow \mathbb{R}$ and $f(x) = x^2$.
\end{flushleft}
\begin{flushleft}
The symbol $\mathbb{R}$ denotes the real numbers.
\end{flushleft}
\begin{flushleft}
If we travel along the x axis left or right, and stop at any point, we can see that there is one function value corresponding to that x coordinate. 
\end{flushleft}
\begin{flushleft}
This is visual evidence that $f(x)$ is a function.
\end{flushleft}
\begin{flushleft}
The graph shows that every element in the domain of $f$ is associated with exactly one element in the range of $f$.
\end{flushleft}
\section{Sequences}
\begin{flushleft}
A sequence is a list of numbers. A sequence can be finite or infinite.
\end{flushleft}
\begin{flushleft}
One example of a sequence is the first ten positive squares.
\end{flushleft}
\begin{align*}
1, 4, 9, 16, 25, 36, 49, 64, 81, 100
\end{align*}
\begin{flushleft}
We can write the sequence of squares symbolically as:
\end{flushleft}
\begin{align*}
a_{n} = n^2 
\end{align*}
\begin{flushleft}
The subscript n can be any natural number. 
\end{flushleft}
\begin{flushleft}
Thus $a_{10} = 10^2 = 100$ is the tenth term in the sequence.
\end{flushleft}
\begin{flushleft}
A sequence is a kind of function. For every input we get exactly one output. For every index we get exactly one value.
\end{flushleft}
\begin{flushleft}
The sequence $a_{n} = n$ is a function $f : \mathbb{N} \rightarrow \mathbb{N}$ defined by $f(n) = n$.
\end{flushleft}
\section{Summation}
\begin{flushleft}
Let's write out the sum of the first ten squares.
\end{flushleft}
\begin{align*}
1^2 + 2^2 + 3^2 + 4^2 + 5^2 + 6^2 + 7^2 + 8^2 + 9^2 + 10^2
\end{align*}
\begin{flushleft}
This takes a lot of time and effort. What if we wanted to sum the first one hundred squares? Is there a concise way of writing this?
\end{flushleft}
\begin{flushleft}
We can write this concisely using summation.
\end{flushleft}
\begin{align*}
\sum_{k=1}^{100} k^2
\end{align*}
\begin{flushleft}
The summation operator $\sum$ gets its symbol from the Greek letter sigma. The summation operator accepts a sequence, a lower bound, and an upper bound, and sums every term in the sequence from the lower bound to the upper bound, up to and including the upper bound.
\end{flushleft}
\begin{align*}
\sum_{k=1}^{10} k^2 = 1^2 + 2^2 + 3^2 + 4^2 + 5^2 + 6^2 + 7^2 + 8^2 + 9^2 + 10^2
\end{align*}
\begin{flushleft}
In the above equation, we see a concise expression on the left hand side, and an expanded expression on the right hand side.
\end{flushleft}
\begin{flushleft}
The summation operator is called a trinary operator because it accepts three operands: a sequence, a lower bound, and an upper bound.
\end{flushleft}
\begin{align*}
\sum_{k=L}^{M} a_{k}
\end{align*}
\begin{flushleft}
In the expression above, $a_{k}$ is the sequence, L is the lower bound of summation, and M is the upper bound of summation. The variable k is the index of summation.
\end{flushleft}
\begin{flushleft}
We can also write the operation using a second notation, shown below.
\end{flushleft}
\begin{align*}
\sum(a_{k}, L, M) 
\end{align*}
\begin{flushleft}
Summation is an operation that adds the terms of a sequence or a subsequence. It can add all of the terms of a sequence, or some of the terms of a sequence. It can add a finite number of terms, or an infinite number of terms. When we use summation on an infinite number of terms, we call it a series.
\end{flushleft}
\begin{flushleft}
We are looking for a formula that sums the first n squares. We can write the sum of the first n squares concisely in this way:
\end{flushleft}
\begin{align*}
\sum_{k=1}^{n} k^2
\end{align*}
\section{The sum of the first n natural numbers}

\begin{lemma}
\begin{equation*}
\sum_{k=1}^{n} k = \frac{n(n+1)}{2}
\end{equation*}
\end{lemma}

\begin{proof}
\begin{flushleft}
Let's write out the sum of the first n natural numbers.
\end{flushleft}

\begin{align*}
\sum_{k=1}^{n} k = 1 + 2 + 3 + ... + n
\end{align*}

\begin{flushleft}
By rearranging terms, we can write the equation in two ways.
\end{flushleft}

\begin{align*}
\sum_{k=1}^{n} k &= 1 + 2 + 3 + ... + n \\
\sum_{k=1}^{n} k &= n + (n-1) + (n-2) + ... + 1
\end{align*}

\begin{flushleft}
Now let's add the two equations, so that 1 is paired with n, 2 is paired with (n-1), 3 is paired with (n-2), and so on.
\end{flushleft}

\begin{align*}
2 * \sum_{k=1}^{n} k = (1 + n) + (2 + (n - 1)) + (3 + (n - 2)) + ... + (n + 1)
\end{align*}

\begin{flushleft}
This gives us n pairs of numbers that sum to $n+1$.
\end{flushleft}

\begin{align*}
2 * \sum_{k=1}^{n} k = n(n + 1)
\end{align*}

\begin{flushleft}
Now let's divide both sides of the equation by 2.
\end{flushleft}

\begin{align*}
\sum_{k=1}^{n} k = \frac{n(1 + n)}{2}
\end{align*}
\end{proof}
\section{Telescoping sums}

\begin{lemma}
Let $a_{k}$ be a sequence of real numbers. Then 
\begin{align*}
\sum_{k=1}^{n} (a_{k} - a_{k-1}) = a_{n} - a_{0}
\end{align*}
\end{lemma}

\begin{proof}
\begin{flushleft}
We can expand the sum, and rearrange the terms, so that every term but the first and the last cancel out.
\end{flushleft}
\begin{align*}
\sum_{k=1}^{n} (a_{k} - a_{k-1}) &= (a_{n} - a_{n-1}) + (a_{n-1} + a_{n-2}) + ... + (a_{1} - a_{0}) \\
&= a_{n} + (-a_{n-1} + a_{n-1}) + (-a_{n-2} + a_{n-2}) + ... + (-a_{1} + a_{1}) - a_{0} \\
&= a_{n} - a_{0}
\end{align*}

\end{proof}

\section{The sum of the first n squares}

\begin{theorem}
\begin{flushleft}
Let $a_{k} = k^2 $ be a sequence of real numbers. Then
\end{flushleft}
\begin{equation*}
\sum_{k=1}^{n} k^2 = \frac{n(n+1)(2n+1)}{6}
\end{equation*}
\end{theorem}

\begin{proof}
\begin{flushleft}
Let $b_{k} = k^3$ be a sequence of real numbers. 
\end{flushleft}

\begin{flushleft}
From lemma 2 we know that
\end{flushleft}

\begin{align*}
\sum_{k=1}^{n} (b_{k} - b_{k-1}) &= b_{n} - b_{0} \\
&= n^3 - 0^3 \\
&= n^3
\end{align*}

\begin{flushleft}
Thus 
\end{flushleft}

\begin{align}
\sum_{k=1}^{n} (b_{k} - b_{k-1}) = n^3
\end{align}

\begin{flushleft}
We also know that 
\end{flushleft}

\begin{align*}
\sum_{k=1}^{n} (b_{k} - b_{k-1}) &= \sum_{k=1}^{n} [k^3 - (k-1)^3] \\
&= \sum_{k=1}^{n} [k^3 - (k^3 - 3k^2 + 3k - 1)] \\
&= \sum_{k=1}^{n} (3k^2 - 3k + 1) \\
&= 3 * \sum_{k=1}^{n} k^2 - 3 * \sum_{k=1}^{n} k + \sum_{k=1}^{n} 1
\end{align*}

\begin{flushleft}
Substituting our result from lemma 1 into the above equation, we get
\end{flushleft}

\begin{align*}
\sum_{k=1}^{n} (b_{k} - b_{k-1}) &= 3 \sum_{k=1}^{n} k^2 - 3 \sum_{k=1}^{n} k + \sum_{k=1}^{n} 1 \\
&= 3 \sum_{k=1}^{n} k^2 - \frac{3n(n+1)}{2} + n \\
&= 3 \sum_{k=1}^{n} k^2 - \frac{3n^2+3n}{2} + \frac{2n}{2} \\
&=  3 \sum_{k=1}^{n} k^2 - \frac{3n^2+n}{2}
\end{align*}

\begin{flushleft}
Thus 
\end{flushleft}

\begin{align}
\sum_{k=1}^{n} (b_{k} - b_{k-1}) = 3 \sum_{k=1}^{n} k^2 - \frac{3n^2+n}{2}
\end{align}

\begin{flushleft}
We can now set equations 1 and 2 equal to each other.
\end{flushleft}

\begin{align*}
3 \sum_{k=1}^{n} k^2 - \frac{3n^2+n}{2} = n^3
\end{align*}

\begin{flushleft}
Adding to both sides of the equation, and simplifying, we get
\end{flushleft}

\begin{align*}
3 \sum_{k=1}^{n} k^2 &= n^3 + \frac{3n^2+n}{2} \\
&= \frac{2n^3 + 3n^2 + n}{2}
\end{align*}

\begin{flushleft}
Dividing both sides of the equation by 3, we get
\end{flushleft}

\begin{align*}
\sum_{k=1}^{n} k^2 &= \frac{2n^3 + 3n^2 + n}{6} \\
&= \frac{n(n+1)(2n+1)}{6}
\end{align*}

\end{proof}
\end{document}