\documentclass{article}
\usepackage{amsmath, amssymb, amsthm}
\newtheorem{definition}{Definition}
\title{Limits}
\author{Andrew Taylor}
\date{February 27 2022}
\begin{document}
\maketitle
\section{Limits}

\begin{definition}
Let $a_{n}$ be a sequence. The limit of $a_{n}$ is A if and only if for all $\epsilon > 0$ there exists a natural number N such that 

\begin{equation*}
|a_{n} - A| < \epsilon
\end{equation*}

for all $n > N$. \\

We write this as 

\begin{equation*}
\lim_{n \to \infty} a_{n} = A
\end{equation*}

and say the sequence $a_{n}$ converges to A.

\end{definition}

\begin{definition}
Let $f$ be a real-valued function and let $a$ be a real number. The limit of $f$ as x approaches a is L if and only if for all $\epsilon > 0$ there exists a $\delta > 0$ such that

\begin{equation*}
|f(x) - L| < \epsilon
\end{equation*}

for all 

\begin{equation*}
0 < |x - a| < \delta
\end{equation*}

We write this as 

\begin{equation*}
\lim_{x \to a} f(x) = L
\end{equation*}

\end{definition}

\end{document}