\documentclass{article}
\usepackage{amsmath, amssymb, amsthm}
\newtheorem{definition}{Definition}
\newtheorem{theorem}{Theorem}
\newtheorem{example}{Example}
\newtheorem{problem}{Problem}
\title{Repeating decimals}
\author{Andrew Taylor}
\date{March 13 2022}
\begin{document}
\maketitle

\begin{definition}
A terminating decimal is a rational number that has a finite number of nonzero digits.
\end{definition}

\begin{example}
The number 1.125 is a terminating decimal. 

\begin{equation*}
1.125 = \frac{10}{8} = \frac{5}{4}
\end{equation*}

\end{example}

\begin{definition}
A repeating decimal is a rational number whose decimal representation repeats in regular cycles.
\end{definition}

\begin{example}
The fraction $\displaystyle \frac{1}{3}$ is a repeating decimal.

\begin{equation*}
\frac{1}{3} = 0.33333... = 0.\overline{3}
\end{equation*}

\end{example}

\begin{example}
The number $\displaystyle \frac{1}{7}$ is a repeating decimal.
\end{example}

\begin{equation*}
\frac{1}{7} = 0.142857142857... = 0.\overline{142857}
\end{equation*}

\begin{theorem}
Every rational number can be written as either a terminating decimal or a repeating decimal.
\end{theorem}

\begin{proof}
Let $\displaystyle \frac{a}{b}$ be a rational number. The division algorithm lets us write the equation 

\begin{equation*}
a = bq + r
\end{equation*}

for unique integers q and r.
\\
\\
We can apply the division algorithm repeatedly, and each time we do, we will get a remainder $r$ from the set \{0, 1, 2, ... b-1\}. Since this set of possible remainders has $b$ elements, we are guaranteed after $b$ applications of the division algorithm to get a cycle or the remainder 0. If we get a cycle that does not end with zero, we have a repeating decimal. If we get zero, we have a terminating decimal. Therefore a rational number is either a terminating decimal or a repeating decimal.
\end{proof}

\begin{problem}
Show that 0.136136136... is a rational number
\end{problem}

\begin{proof}
Let $x = 0.136136136...$ 
\\
\\
Then $1000x = 136.136136...$. 
\\
\\
Subtracting, we get:

\begin{align*}
1000x - x &= 136 \\
999x &= 136 \\
x &= \frac{136}{999}
\end{align*}

Therefore $x = \displaystyle \frac{136}{999}$ and $x$ is a rational number.
\end{proof}

\end{document}