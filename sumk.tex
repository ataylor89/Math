\documentclass{article}
\usepackage{amsmath}
\begin{document}
\section{Find a formula for $\sum_{k=1}^{n} k$}
\begin{flushleft}
First let's look at the binomial expansion of $(k-1)^2$:
\end{flushleft}
\begin{align}
(k-1)^2 = k^2 - 2k + 1 
\end{align}
\begin{flushleft}
Let $a_{k} = k^2$. Let's create the telescoping sequence $t_{k} = a_{k} - a_{k-1}$.
\end{flushleft}
\begin{align*}
t_{k} &= a_{k}-a_{k-1} \\
&= k^2 - (k-1)^2 \\
&= k^2 - (k^2 - 2k + 1) \\
&= 2k -1
\end{align*}
\begin{flushleft}
Thus we have the equation:
\end{flushleft}
\begin{align}
t_{k} = 2k - 1 
\end{align}
\begin{flushleft}
We can sum both sides of equation (2) to get a telescoping series.
\end{flushleft}
\begin{align}
\sum_{k=1}^{n} t_{k} = \sum_{k=1}^{n} (2k-1)
\end{align}
\begin{flushleft}
All terms in the series $\sum_{k=1}^{n} t_{k}$ cancel out, except for the first and the last:
\end{flushleft}
\begin{align*}
\sum_{k=1}^{n} t_{k} &= \sum_{k=1}^{n} (a_{k} - a_{k-1}) \\
&= a_{n} - a_{0} \\
&= n^2 - 0^2 \\
&= n^2
\end{align*}
\begin{flushleft}
Now we have the equation:
\end{flushleft}
\begin{align}
\sum_{k=1}^{n} t_{k} = n^2
\end{align}
\begin{flushleft}
We can perform substitution in equation (3)
\end{flushleft}
\begin{align*}
n^2 &= \sum_{k=1}^{n} (2k-1) \\
       &= \sum_{k=1}^{n} 2k - \sum_{k=1}^{n} 1 \\
       &= 2 * \sum_{k=1}^{n} k - \sum_{k=1}^{n} 1
\end{align*}
\begin{flushleft}
Thus we have the equation:
\end{flushleft}
\begin{align}
n^2 = 2 * \sum_{k=1}^{n} k - \sum_{k=1}^{n} 1
\end{align}
\begin{flushleft}
We can solve equation (5) for $\sum_{k=1}^{n} k$.
\end{flushleft}
\begin{align*}
2 *  \sum_{k=1}^{n} k &= n^2 + \sum_{k=1}^{n} 1 \\
&= n^2 + n \\
&= n(n+1)
\end{align*}
\begin{flushleft}
And we get the formula:
\end{flushleft}
\begin{align}
\sum_{k=1}^{n} k = \frac{n(n+1)}{2}
\end{align}
\section{Find a formula for $\sum_{k=1}^{n} k^2$}
\begin{flushleft}
We can apply the same technique as before.
\end{flushleft}
\begin{flushleft}
Let's look at the binomial expansion of $(k-1)^3$.
\end{flushleft}
\begin{align*}
(k-1)^3 &= (k-1)(k^2 - 2k + 1) \\
&= k(k^2 - 2k + 1) - (k^2 - 2k + 1) \\
&= k^3 - 2k^2 + k - k^2 + 2k - 1 \\
&= k^3 - 3k^2 + 3k - 1
\end{align*}
\begin{flushleft}
The coefficients are what we expect from Pascal's triangle.
\end{flushleft}
\begin{flushleft}
This gives us the equation: 
\end{flushleft}
\begin{align}
(k-1)^3 = k^3 - 3k^2 + 3k - 1
\end{align}
\begin{flushleft}
Now let's look at the general term of a telescoping series.
\end{flushleft}
\begin{flushleft}
Let $t_{k} = k^3 - (k-1)^3$
\end{flushleft}
\begin{align*}
t_{k} &= k^3 - (k-1)^3 \\
&= k^3 - (k^3 - 3k^2 + 3k - 1) \\
&= 3k^2 - 3k + 1
\end{align*}
\begin{flushleft}
Thus we have the equation:
\end{flushleft}
\begin{align}
t_{k} = 3k^2 - 3k + 1
\end{align}
\begin{flushleft}
The left hand side of the equation is the term of a sequence. The right hand side is a polynomial. We can take the summation of both sides of the equation.
\end{flushleft}
\begin{align}
\sum_{k=1}^{n} t_{k} = \sum_{k=1}^{n} (3k^2 - 3k + 1)
\end{align}
\begin{flushleft}
The left hand side of the equation telescopes.
\end{flushleft}
\begin{align*}
n^3 &= \sum_{k=1}^{n} (3k^2 - 3k + 1) \\
&= \sum_{k=1}^{n} (3k^2) - \sum_{k=1}^{n} 3k + \sum_{k=1}^{n} 1 \\
&= 3 * \sum_{k=1}^{n} k^2 - 3 * \sum_{k=1}^{n} k + n
\end{align*}
\begin{flushleft}
We can simplify this equation more using the result from section 1.
\end{flushleft}
\begin{align*}
n^3 =  3 * \sum_{k=1}^{n} k^2 - \frac{3n(n+1)}{2} + n
\end{align*}
\begin{flushleft}
Rearranging we get:
\end{flushleft}
\begin{align*}
3 * \sum_{k=1}^{n} k^2 = n^3 - n + \frac{3n(n+1)}{2} 
\end{align*}
\begin{flushleft}
Let's give the right hand side a common denominator, and simplify.
\end{flushleft}
\begin{align*}
3 * \sum_{k=1}^{n} k^2 &= \frac{2n^3 - 2n + 3n(n+1)}{2} \\
&= \frac{2n^3 - 2n + 3n^2 + 3n}{2} \\
&= \frac{2n^3 + 3n^2 + n}{2} \\
&= \frac{n(2n^2 + 3n + 1)}{2} \\
&= \frac{n(n+1)(2n+1)}{2} 
\end{align*}
\begin{flushleft}
Now we can divide both sides of the equation by 3.
\end{flushleft}
\begin{flushleft}
This gives us the formula for the sum of squares:
\end{flushleft}
\begin{align}
\sum_{k=1}^{n} k^2 = \frac{n(n+1)(2n+1)}{6}
\end{align}
\begin{flushleft}
The same technique we used in the first section (telescoping) helps us find a formula for the sum of squares. We can use this technique to find a formula for the sum of cubes, fourths, and so on.
\end{flushleft}
\end{document}