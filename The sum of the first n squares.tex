\documentclass{article}

\usepackage{amsmath, amssymb, amsthm}

\title{The sum of the first n squares}
\author{Andrew Taylor}
\date{August 25 2022}

\begin{document}
\maketitle

In this paper we will find a formula for the sum of the first n squares: $$ \sum_{k=0}^{n} k^{2} $$

The trick to solving this problem is writing $n^3$ as a series. \\

We can do this using a well-known identity. \\

Let $a_n$ be any real-valued sequence. Then

\begin{equation*}
a_{n} - a_{0} = \sum_{k=1}^{n} \left( a_{k} - a_{k-1} \right)
\end{equation*}

The expression on the right is called a telescoping series, because it collapses like a folding telescope. \\

(When a folding telescope collapses, you are left with the front part and the back part joined together. When a telescoping series collapses, you are left with the first and last terms of the series.) \\

Now we can write $n^3$ as a telescoping series, using the sequence $a_{n} = n^{3}$.

\begin{align*}
n^3 &= n^3 - 0^3 \\
&= \sum_{k=0}^{n} \left( k^3 - (k-1)^3 \right)
\end{align*}

Expanding the right side of the equation (we can calculate $(k-1)^3$ quickly using Pascal's triangle) we get:

\begin{align*}
n^3 &= \sum_{k=0}^{n} \left( k^3 - (k-1)^3 \right) \\
&= \sum_{k=0}^{n} \left( k^3 - (k^3 - 3k^2 + 3k - 1) \right) \\
&= \sum_{k=0}^{n} \left( k^3 - k^3 + 3k^2 - 3k + 1 \right) \\
&= \sum_{k=0}^{n} \left( 3k^2 - 3k + 1 \right) 
\end{align*}

Using the laws for series, we can write each term on the right hand side as its own series, and factor out the constants. \\

\begin{align*}
n^3 &= \sum_{k=0}^{n} \left( 3k^2 - 3k + 1) \right) \\
&= \sum_{k=0}^{n} 3k^2 - \sum_{k=0}^{n} 3k + \sum_{k=0}^{n} 1 \\
&= 3 \sum_{k=0}^{n} k^2 - 3 \sum_{k=0}^{n} k + \sum_{k=0}^{n} 1 \\
&= 3 \sum_{k=0}^{n} k^2 - \dfrac{3n(n+1)}{2} + n \\
\end{align*}

Now we can solve for the series $$\sum_{k=0}^{n} k^2$$.

\begin{align*}
n^3 &= 3 \sum_{k=0}^{n} k^2 - \dfrac{3n(n+1)}{2} + n \\
n^3 + \dfrac{3n(n+1)}{2} - n &= 3 \sum_{k=0}^{n} k^2 \\
\dfrac{2n^3 + 3n(n+1) - 2n}{2} &= 3 \sum_{k=0}^{n} k^2 \\
\dfrac{2n^3 + 3n(n+1) - 2n}{6} &= \sum_{k=0}^{n} k^2 \\
\dfrac{n(2n^2 + 3(n+1) - 2)}{6} &= \sum_{k=0}^{n} k^2 \\
\dfrac{n(2n^2 + 3n + 1)}{6} &= \sum_{k=0}^{n} k^2 \\
\dfrac{n(n+1)(2n+1)}{6} &= \sum_{k=0}^{n} k^2
\end{align*}

After factoring the lefthand side, we arrive at the identity $$\sum_{k=0}^{n} k^2 = \dfrac{n(n+1)(2n+1)}{6} $$

In summary, we solved this problem by using a trick: we wrote $n^3$ as a telescoping series.

\begin{align*}
n^3 &= n^3 - 0^3 \\
&= \sum_{k=0}^{n} \left( k^3 - (k-1)^3 \right)
\end{align*}

Expanding the right hand side of the equation, we get the term we are looking for. $$ n^3 = \sum_{k=0}^{n} \left( 3k^2 - 3k + 1 \right)  $$

The term we are looking for is $$ \sum_{k=0}^{n} k^2 $$

Now we just have to solve for this term. \\

Doing so gives us the identity $$\sum_{k=0}^{n} k^2 = \dfrac{n(n+1)(2n+1)}{6} $$

\end{document}